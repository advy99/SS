\documentclass[12pt, spanish]{article}
\usepackage[spanish]{babel}
\selectlanguage{spanish}
%\usepackage{natbib}
\usepackage{url}
\usepackage[utf8x]{inputenc}
\usepackage{graphicx}
\graphicspath{{images/}}
\usepackage{parskip}
\usepackage{fancyhdr}
\usepackage{vmargin}
\usepackage{multirow}
\usepackage{float}
\usepackage{chngpage}
\usepackage{enumitem}

\usepackage{amsfonts}

\usepackage{subcaption}

\usepackage{hyperref}
\usepackage[
    type={CC},
    modifier={by-nc-sa},
    version={4.0},
]{doclicense}

\hypersetup{
    colorlinks=true,
    linkcolor=blue,
    filecolor=magenta,
    urlcolor=cyan,
}

% para codigo
\usepackage{listings}
\usepackage{xcolor}



%% configuración de listings

\definecolor{listing-background}{HTML}{F7F7F7}
\definecolor{listing-rule}{HTML}{B3B2B3}
\definecolor{listing-numbers}{HTML}{B3B2B3}
\definecolor{listing-text-color}{HTML}{000000}
\definecolor{listing-keyword}{HTML}{435489}
\definecolor{listing-identifier}{HTML}{435489}
\definecolor{listing-string}{HTML}{00999A}
\definecolor{listing-comment}{HTML}{8E8E8E}
\definecolor{listing-javadoc-comment}{HTML}{006CA9}

\lstdefinestyle{eisvogel_listing_style}{
  language         = python,
%$if(listings-disable-line-numbers)$
%  xleftmargin      = 0.6em,
%  framexleftmargin = 0.4em,
%$else$
  numbers          = left,
  xleftmargin      = 0em,
 framexleftmargin = 0em,
%$endif$
  backgroundcolor  = \color{listing-background},
  basicstyle       = \color{listing-text-color}\small\ttfamily{}\linespread{1.15}, % print whole listing small
  breaklines       = true,
  frame            = single,
  framesep         = 0.19em,
  rulecolor        = \color{listing-rule},
  frameround       = ffff,
  tabsize          = 4,
  numberstyle      = \color{listing-numbers},
  aboveskip        = 1.0em,
  belowskip        = 0.1em,
  abovecaptionskip = 0em,
  belowcaptionskip = 1.0em,
  keywordstyle     = \color{listing-keyword}\bfseries,
  classoffset      = 0,
  sensitive        = true,
  identifierstyle  = \color{listing-identifier},
  commentstyle     = \color{listing-comment},
  morecomment      = [s][\color{listing-javadoc-comment}]{/**}{*/},
  stringstyle      = \color{listing-string},
  showstringspaces = false,
  escapeinside     = {/*@}{@*/}, % Allow LaTeX inside these special comments
  literate         =
  {á}{{\'a}}1 {é}{{\'e}}1 {í}{{\'i}}1 {ó}{{\'o}}1 {ú}{{\'u}}1
  {Á}{{\'A}}1 {É}{{\'E}}1 {Í}{{\'I}}1 {Ó}{{\'O}}1 {Ú}{{\'U}}1
  {à}{{\`a}}1 {è}{{\'e}}1 {ì}{{\`i}}1 {ò}{{\`o}}1 {ù}{{\`u}}1
  {À}{{\`A}}1 {È}{{\'E}}1 {Ì}{{\`I}}1 {Ò}{{\`O}}1 {Ù}{{\`U}}1
  {ä}{{\"a}}1 {ë}{{\"e}}1 {ï}{{\"i}}1 {ö}{{\"o}}1 {ü}{{\"u}}1
  {Ä}{{\"A}}1 {Ë}{{\"E}}1 {Ï}{{\"I}}1 {Ö}{{\"O}}1 {Ü}{{\"U}}1
  {â}{{\^a}}1 {ê}{{\^e}}1 {î}{{\^i}}1 {ô}{{\^o}}1 {û}{{\^u}}1
  {Â}{{\^A}}1 {Ê}{{\^E}}1 {Î}{{\^I}}1 {Ô}{{\^O}}1 {Û}{{\^U}}1
  {œ}{{\oe}}1 {Œ}{{\OE}}1 {æ}{{\ae}}1 {Æ}{{\AE}}1 {ß}{{\ss}}1
  {ç}{{\c c}}1 {Ç}{{\c C}}1 {ø}{{\o}}1 {å}{{\r a}}1 {Å}{{\r A}}1
  {€}{{\EUR}}1 {£}{{\pounds}}1 {«}{{\guillemotleft}}1
  {»}{{\guillemotright}}1 {ñ}{{\~n}}1 {Ñ}{{\~N}}1 {¿}{{?`}}1
  {…}{{\ldots}}1 {≥}{{>=}}1 {≤}{{<=}}1 {„}{{\glqq}}1 {“}{{\grqq}}1
  {”}{{''}}1
}
\lstset{style=eisvogel_listing_style}


\usepackage[default]{sourcesanspro}

\setmarginsrb{2 cm}{1 cm}{2 cm}{2 cm}{1 cm}{1.5 cm}{1 cm}{1.5 cm}

\title{Práctica 3:\\
Modelos de Simulación Dinámicos y Discretos.\hspace{0.05cm} }
\author{Antonio David Villegas Yeguas}
\date{\today}

\renewcommand*\contentsname{hola}

\makeatletter
\let\thetitle\@title
\let\theauthor\@author
\let\thedate\@date
\makeatother

\pagestyle{fancy}
\fancyhf{}
\rhead{\theauthor}
\lhead{\thetitle}
\cfoot{\thepage}

\begin{document}

%%%%%%%%%%%%%%%%%%%%%%%%%%%%%%%%%%%%%%%%%%%%%%%%%%%%%%%%%%%%%%%%%%%%%%%%%%%%%%%%%%%%%%%%%

\begin{titlepage}
    \centering
    \vspace*{0.3 cm}
    \includegraphics[scale = 0.50]{ugr.png}\\[0.7 cm]
    %\textsc{\LARGE Universidad de Granada}\\[2.0 cm]
    \textsc{\large 4º CSI 2020/21 - Grupo 1}\\[0.5 cm]
    \textsc{\large Grado en Ingeniería Informática}\\[0.5 cm]
    \rule{\linewidth}{0.2 mm} \\[0.2 cm]
    { \huge \bfseries \thetitle}\\
    \rule{\linewidth}{0.2 mm} \\[1 cm]

    \begin{minipage}{0.4\textwidth}
        \begin{flushleft} \large
            \emph{Autor:}\\
            \theauthor\\
			 \emph{DNI:}\\
            77021623-M
            \end{flushleft}
            \end{minipage}~
            \begin{minipage}{0.4\textwidth}
            \begin{flushright} \large
            \emph{Asignatura: \\
            Simulación de Sistemas}   \\
            \emph{Correo:}\\
            advy99@correo.ugr.es
        \end{flushright}
    \end{minipage}\\[0.5cm]

    {\large \thedate}\\[0.5cm]
    %{\url{https://github.com/advy99/AA/}}
    {\doclicenseThis}

    \vfill

\end{titlepage}

%%%%%%%%%%%%%%%%%%%%%%%%%%%%%%%%%%%%%%%%%%%%%%%%%%%%%%%%%%%%%%%%%%%%%%%%%%%%%%%%%%%%%%%%%

\tableofcontents
\pagebreak

%%%%%%%%%%%%%%%%%%%%%%%%%%%%%%%%%%%%%%%%%%%%%%%%%%%%%%%%%%%%%%%%%%%%%%%%%%%%%%%%%%%%%%%%%


\section*{Introducción}

Para esta práctica se nos plantean distintos modelos de simulación de cara a conocer las diferencias entre una gestión de tiempo de incremento fijo o variable, así como estudiar el funcionamiento de modelos dinámicos y discretos.

\section{Gestión del tiempo: Métodos de incremento fijo e incremento variable de tiempo}

Este sistema necesita que cierto número de máquinas estén siempre en funcionamiento, de forma que para evitar fallos se dispone de una serie de repuestos. Si una máquina se estropea, se cambia por un repuesto, y se envia a un taller para que la máquina sea reparada, de forma que cuando esté reparada, pasará a formar parte de las máquinas de repuesto.

De esta forma, este sistema se basa en gestionar los estados en los que falla una máquina, se repara, y vuelve de la reparación, de forma que tendremos como parámetros el número de máquinas, el número de reparadores del taller, el tiempo de reparación y el tiempo de fallo, estos dos últimos siguiento una distribución de probabilidad exponencial.


\subsection{Prueba de los dos modelos implementados}

Tras desarrollar ambos modelos he realizado las pruebas pedidas en el guión de prácticas, de cara a comparar las ejecuciones, obteniendo los siguientes resultados:

\begin{table}[H]
\begin{tabular}{|c|c|c|c|c|c|}
\hline
\multicolumn{6}{|c|}{\textbf{Sistema de tiempo fijo}}                                          \\ \hline
T. reparacion & T. fallo & D. fallos & N. fallos & D. media fallos & D. media fallos teorica \\ \hline
0.5           & 1        & 0         & 4091      & 0               & 0                       \\ \hline
2             & 1        & 4441      & 2905      & 1.52874         & 2                       \\ \hline
12            & 24       & 1251      & 113       & 11.0708         & 12.0325                 \\ \hline
48            & 24       & 5559      & 107       & 51.9533         & 48.3043                 \\ \hline
720           & 1440     & 584       & 1         & 584             & 833.25                  \\ \hline
2880          & 1440     & 1118      & 1         & 1118            & 3333							\\ \hline
\end{tabular}
\end{table}

\begin{table}[H]
\begin{tabular}{|c|c|c|c|c|c|}
\hline
\multicolumn{6}{|c|}{\textbf{Sistema de tiempo variable}}                                      \\ \hline
T. reparacion & T. fallo & D. fallos & N. fallos & D. media fallos & D. media fallos teorica \\ \hline
0.5           & 1        & 0         & 0         & 0               & 0                       \\ \hline
2             & 1        & 5684.44   & 2864      & 1.98479         & 2                       \\ \hline
12            & 24       & 1297.14   & 105       & 12.3537         & 12                      \\ \hline
48            & 24       & 5570.06   & 110       & 50.637          & 48                      \\ \hline
720           & 1440     & 584.044   & 1         & 584.044         & 720                     \\ \hline
2880          & 1440     & 1118.76   & 2         & 559.381         & 2880							\\ \hline
\end{tabular}
\end{table}


Vemos como en general el sistema que utiliza la variación de tiempo variable se ajusta mejor a los resultados que teoricamente deberíamos obtener, ya que a diferencia del sistema de incremento fijo, se mueve al punto exacto donde ocurre el suceso, mientras que si en el sistema de tiempo fijo ocurre un suceso en el intervalo en el que no se modifica el reloj, no se gestiona hasta el siguiente avance del reloj.

También vemos que si utilizamos medidas más precisas de tiempo, este problema se va solucionando, y es que el espacio de muestreo del reloj fijo es mayor, evitando el problema, ya que, por ejemplo, si usamos un día, pasará de un día a otro, cuando un evento puede ocurrir al inicio del día, mientras que al usar horas o minutos, puede gestionar el suceso en ese espacio muestral más amplio.


\subsection{Comparación de eficiencia entre ambos modelos}

Por último, comprobaremos que sistema es más rápido a la hora de la ejecución. Probando con un gran número de días se ha obtenido lo siguiente:

\begin{figure}[H]
  \centering
      \includegraphics[width=\textwidth]{t1_eficiencia.png}
 		\caption{Comparación de tiempo de ejecución entre el sistema que usa tiempo fijo contra el de tiempo variable.}
\end{figure}


Vemos que el sistema de tiempo fijo, como era de esperar, se comporta peor que el sistema de tiempo variable, y es que el sistema de tiempo fijo en el tiempo que no tiene ningún evento tiene que seguir actualizando su reloj, mientras que el sistema de tiempo variable salta al siguiente evento directamente, evitandose esos ciclos que unicamente incrementa el reloj.


\section{Sucesos y grafos de sucesos: Estructura de un programa de simulación dinámico y discreto}

En este apartado trabajaremos con un sistema similar al anterior, pero en este caso nos centraremos en como funciona su grafo de sucesos, así como en sus parámetros óptimos.

Para este modelo, el número de máquinas y repuestos será variable y nos centraremos en probar distintas configuraciones de cara a comparar si es mejor un sistema con varios reparadores o un único reparador.


Para realizar esta tarea he fijado el número de máquinas a diez y el número de repuestos a dos, es decir, en total doce máquinas, mientras que el tiempo de reparación será de un día y seis horas, y el tiempo de fallo será de un día.


% Please add the following required packages to your document preamble:
% \usepackage{graphicx}
\begin{table}[H]
\centering
\resizebox{\textwidth}{!}{%
\begin{tabular}{|c|c|c|c|c|c|}
\hline
\multicolumn{6}{|c|}{\textbf{Comparación del sistema variando el número de reparadores}}                                                                                                                                                                                          \\ \hline
\textbf{N. reparadores} & \textbf{D. media fallos} & \textbf{T. medio entre fallos} & \textbf{\begin{tabular}[c]{@{}c@{}}N. medio maquinas \\ en reparación\end{tabular}} & \textbf{P. tiempo ocio} & \textbf{\begin{tabular}[c]{@{}c@{}}P. duración total\\ fallos\end{tabular}} \\ \hline
1                       & 364.638                  & 0                              & 11.140                                                                              & 0.038                   & 99.901                                                                      \\ \hline
2                       & 364.868                  & 0                              & 10.250                                                                              & 0.011                   & 99.964                                                                      \\ \hline
3                       & 364.868                  & 0                              & 9.526                                                                               & 0.019                   & 99.964                                                                      \\ \hline
4                       & 90.872                   & 0.460                          & 8.622                                                                               & 0.339                   & 99.595                                                                      \\ \hline
5                       & 24.181                   & 0.154                          & 7.760                                                                               & 1.708                   & 99.374                                                                      \\ \hline
6                       & 14.459                   & 0.141                          & 7.302                                                                               & 5.431                   & 99.034                                                                      \\ \hline
7                       & 16.449                   & 0.142                          & 6.677                                                                               & 12.531                  & 99.147                                                                      \\ \hline
8                       & 7.624                    & 0.142                          & 6.548                                                                               & 21.254                  & 98.177                                                                      \\ \hline
9                       & 10.908                   & 0.151                          & 6.412                                                                               & 29.160                  & 98.620                                                                      \\ \hline
10                      & 12.944                   & 0.091                          & 6.536                                                                               & 34.707                  & 99.293                                                                      \\ \hline
\end{tabular}%
}
\end{table}


Vemos como en este caso, aunque se mantega un gran porcentaje de ocupación por parte de los reparadores al tener los fallos con menor tiempo que lo que suelen tardar en reparar una máquina, el tener más reparadores permite que el tiempo que se tiene de media con fallos baje desde la totalidad de la simulación, a solo unos cuantos días, por lo que los sistemas no resultan equivalentes, y con esto estamos demostrando la gran ayuda que aportan los reparadores, y que estos estén en un gran número si se tarda más en reparar una máquina que en lo que suele tardar en fallar.

Aun así vemos que ya en los últimos casos donde hay muchos reparadores, aunque estén ocupados depende de la aleatoriedad de como se rompan las máquinas, lo que hace que en el tiempo medio de fallos pueda parecer que al final sube, pero en realidad es por la aleatoriedad del sistema. Como vemos, esto no pasa en el porcentaje de tiempo libre, que siempre se incrementa, al haber más reparadores.


\begin{figure}[H]
  \centering
      \includegraphics[width=\textwidth]{tlibre_reparadores.png}
 		\caption{Porcentaje de tiempo libre de reparadores según el número de reparadores.}
\end{figure}



Por último para este apartado, como nos pide el guión, la configuración con la que conseguimos mantener el porcentaje de duración total de fallos por debajo de un 10\% es la siguiente:

\begin{lstlisting}
Con 1 repuestos y 3 reparadores se ha conseguido que el porcentaje total de fallos esté por debajo de un 89.382 por ciento del tiempo
\end{lstlisting}

Además, en el script adjunto a la práctica aparecerán más configuraciones que consiguen rebajar aún más este porcentaje.

\subsection{Modificación del modelo original}

También se nos pide realizar una modificación al sistema, de forma que se añada un suceso de mantenimiento (con su respectivo fin de mantenimiento), de cara a que si hay algún reparador libre, revise las máquinas para que tarden más en estropearse.

De cara a desarrollar y comprobar este apartado simplemente se han añadido dichos estados al código, con sus respectivas funciones que realizan lo indicado en el grafo de sucesos del guión, y se ha ejecutado con los mismos parámetros que el apartado anterior, añadiendo un tiempo de revisión de un día, y un tiempo de reparación uniforme entre dos horas y media y veinte horas.

Con esto obtenemos los siguientes resultados:

\begin{lstlisting}
Con 1 repuestos y 2 reparadores se ha conseguido que el porcentaje total de fallos esté por debajo de un 88.890 por ciento
\end{lstlisting}

Que como vemos, ha mejorado el anterior, necesitando un reparador menos. Además tambien podemos ver que con la misma cantidad de reparadores, se obtiene un resultado mucho mejor:

\begin{lstlisting}
Con 1 repuestos y 3 reparadores se ha conseguido que el porcentaje total de fallos esté por debajo de un 38.880 por ciento
\end{lstlisting}

\newpage

\section{Mi tercer modelo de simulación discreto}

En este apartado trabajaremos con un sistema de simulación de un puerto. Dicho puerto dispone de un remolcador que puede remolcar hasta tres petroleros y es el encargado de llevar los barcos al puerto.

Aunque el sistema ya se nos proporciona con la implementación correspondiente al grafo que aparece en el guión de prácticas se nos pide implementar ciertas modificaciones.

\subsection{Aumentar los puntos de atraque}

En principio el remolcador puede cargar con tres petroleros. Se nos pide que este número sea variable, entre tres, cuatro y cinco. Para implementar este cambio simplemente he modificado el programa, de forma que se proporcione como parámetro el número de puntos de atraque.

\subsection{Remolcador al que no le afectan las tormentas}

Como nos explica el guión, si durante la operación del remolcador se produce una tormenta, este tiene que volver al puerto, esperar a que acabe la tormenta, y seguir operando una vez pase. Esta modificación se basa en que el remolcador no tiene problema en operarar durante una tormenta.

Para implementar esta modifiación simplemente he añadido un parámetro de entrada, que si vale cero no le afectan las tormentas, y si es distinto de cero si que le afectan. De cara a que no le afecten las tormentas, en la inicialización, al generar el primer suceso de tormenta, si no le afectan las tormentas este suceso no se añade, de forma que nunca va a ocurrir una tormenta (ni por lo tanto fin de tormenta, ya que este estado era creado por la tormenta).

\subsection{Remolcador más rápido}

Esta cambio simplemente es modificar la velocidad del remolcador. Estaba fijado en el programa, lo he añadido como parámetro de entrada de cara a controlarlo.

\begin{table}[H]
\centering
\begin{tabular}{|c|c|c|}
\hline
\multicolumn{3}{|c|}{\textbf{Comparación de las tres modificaciones}}                                                                                                                                                                                  \\ \hline
\textbf{Modificación}                                             & \textbf{\begin{tabular}[c]{@{}c@{}}N. medio barcos \\ en cola de llegadas\end{tabular}} & \textbf{\begin{tabular}[c]{@{}c@{}}N. medio barcos \\ en la cola de salida\end{tabular}} \\ \hline
3 atraques                                                        & 1.542977                                                                                & 0.033080                                                                                 \\ \hline
4 atraques                                                        & 0.088477                                                                                & 0.027314                                                                                 \\ \hline
5 atraques                                                        & 0.048014                                                                                & 0.027638                                                                                 \\ \hline
\begin{tabular}[c]{@{}c@{}}No le afectan\\ tormentas\end{tabular} & 0.694968                                                                                & 0.009782                                                                                 \\ \hline
\begin{tabular}[c]{@{}c@{}}Remolcador\\ más rápido\end{tabular}   & 0.931084                                                                                & 0.028482                                                                                 \\ \hline
\end{tabular}
\end{table}


Vemos como la mejor modificación es la de tener un mayor número de puntos de atraque, ya que permite vaciar rápidamente la cola de entrada, de forma que no haya barcos esperando, seguida de la posibilidad de que no le afecten las tormentas, permitiendo que aun con menos puntos de atraque pueda utilizar el tiempo de las tormentas, y finalmente el que el barco sea más rápido, ya que aunque sea más rápido, sigue sin poder cargar tantos petroleros y le siguen afectando las tormentas. Aun así, todas las mejoras (como cabia esperar), son preferibles al sistema original.


Por último, para realizar la comparación, se nos pide utilizar una nueva medida de rendimiento. Dependiendo del tipo de petrolero, tendrá asociado un numero de toneladas cargadas. Debemos investigar cual de las tres alternativas es mejor desde este punto de vista.

\begin{table}[H]
\centering
\begin{tabular}{|c|c|}
\hline
\multicolumn{2}{|c|}{\textbf{Comparación de las tres modificaciones}}                                                                                       \\ \hline
\textbf{Modificación}                                             & \textbf{\begin{tabular}[c]{@{}c@{}}Carga total transportada\end{tabular}} \\ \hline
3 atraques                                                        & 1814000                                                                                 \\ \hline
4 atraques                                                        & 1873000                                                                                 \\ \hline
5 atraques                                                        & 1766000                                                                                 \\ \hline
\begin{tabular}[c]{@{}c@{}}No le afectan\\ tormentas\end{tabular} & 1761000                                                                                 \\ \hline
\begin{tabular}[c]{@{}c@{}}Remolcador\\ más rápido\end{tabular}   & 1777000                                                                                 \\ \hline
\end{tabular}
\end{table}

En este caso, debido a que los tipos de barcos son aleatorios y no se sabe a priori que tipo de barco es, esta medida no es muy fiable, ya que como hemos visto antes, aunque haya menos barcos esperando, dependiendo del tipo de barco se sumará un peso u otro y esto es totalmente aleatorio.


\newpage

\section{Análisis de salidas y experimentación}

En este último apartado se nos pide analizar las salidas para experimentar el número de simulaciones necesarias.

\subsection{Número de simulaciones}

Hasta este punto, todas las ejecuciones del puerto se han ejecutado con una simulación. Ahora vamos a comprobar la importancia de realizar distintas simulaciones, y para eso vamos a comparar el modelo original con el modelo con el remolcador al que no afectan las tormentas.

Para realizar esta comparación, ejecutaremos ambos modelos cien veces, y obtendremos un porcentaje de veces que es preferible un modelo frente a otro con 1, 5, 10, 25 y 50 simulaciones en cada ejecución, obteniendo los siguientes resultados:

\begin{table}[H]
\centering
\begin{tabular}{|c|c|c|}
\hline
\multicolumn{3}{|c|}{\textbf{Comparación de los sistemas A y B}}                                  \\ \hline
\textbf{Simulaciones} & \textbf{Porcentaje A es preferible} & \textbf{Porcentaje B es preferible} \\ \hline
1                     & 39                                  & 61                                  \\ \hline
5                     & 14                                  & 86                                  \\ \hline
10                    & 20                                  & 80                                  \\ \hline
25                    & 4                                   & 96                                  \\ \hline
50                    & 2                                   & 98                                  \\ \hline
\end{tabular}
\end{table}


Como vemos, el número de simulaciones es muy importante, ya que al ser modelos probabilísticos puede ser que en ciertas simulaciones no salga el resultado esperado.


También se nos pide comparar el sistema A y C, obteniendo los siguientes resultados:

\begin{table}[H]
\centering
\begin{tabular}{|c|c|c|}
\hline
\multicolumn{3}{|c|}{\textbf{Comparación de los sistemas A y C}}                                  \\ \hline
\textbf{Simulaciones} & \textbf{Porcentaje A es preferible} & \textbf{Porcentaje C es preferible} \\ \hline
1                     & 46                                  & 54                                  \\ \hline
5                     & 42                                  & 58                                  \\ \hline
10                    & 56                                  & 44                                  \\ \hline
25                    & 49                                  & 51                                  \\ \hline
50                    & 41                                  & 59                                  \\ \hline
\end{tabular}
\end{table}

En este caso no vemos la distinción tan clara, y es que como vimos en apartados anteriores, estos sistemas erán más similares y no existía tanta diferencia.

% \begin{thebibliography}{9}
%
%
% \end{thebibliography}

\end{document}
